%-------------------------------------------------------------------------------
% File:		    Presentation.tex
% Author:	    Igor Janjic, Danny Duangphachanh, Daniel Friedman
% Description:	[ECE 4564] Network Applications Design
%		        Project proposal presentation.
%%------------------------------------------------------------------------------ 

% Class options include: notes, notesonly, handout, trans, hidesubsections,
% shadesubsections, inrow, blue, red, grey, brown.
\documentclass[]{beamer}


% Theme for beamer presentation.
% Other themes include: beamerthemebars, beamerthemelined, beamerthemetree,
% beamerthemetreebars. 
\usepackage{beamerthemesplit}
\usepackage{caption}

\title{PiFeed - Feed Your Pets with a Raspberry Pi!}
\author{Daniel Friedman :: Igor Janjic :: Daniel Duangphachanh}
\institute{Virginia Tech}
\date{\today}

\begin{document}

% Creates title page of slide show using above information.
\begin{frame}
  \titlepage
\end{frame}
\note{}

%\section[Outline]{}
%\begin{frame}
%	\frametitle{Outline}
%	\tableofcontents
%\end{frame}

\section{Objective}
\begin{frame}
	\frametitle{Objective}
    The objectives of Assignment 2 were to:
    \begin{itemize}
    	\item Understand the publish/subscribe message model
    	\begin{itemize}
        	\item Sending messages to a message broker
            \item Creating message topics
            \item Getting messages from a message broker
            \item Configuring a message exchange
            \item Creating queues
            \item Binding a queue to an exchange
            \item Subscribing to a particular ‘topic’
            \item Properly cleaning up a queue upon disconnect
        \end{itemize}
       
    	\item Understand Rabbitmq implementation of AMQP protocol
        \item Effectively work on a team of two
	\end{itemize}
\end{frame}



\section{Approach}
\subsection{Utilization Service}
\begin{frame}
	\frametitle{Utilization Service}
    The utilization service was written following the example code.
    \begin{itemize}
    	\item Thing 1
    	\item Thing 2
    	\item Thing 3
	\end{itemize}
\end{frame}


\subsection{Stats Client}
\begin{frame}
	\frametitle{Stats Client}
    The stats client was written following the example code.
    \begin{itemize}
    	\item Connect to the message broker
    	\item Declare the pi\_utilization direct exchange (create the exchange if it has not been created)
    	\item Declare an exclusive queue (create the queue)
        \item Bind the queue to the pi\_utilization exchange filtering on the routing key
        \item Consume messages
	\end{itemize}
\end{frame}

\subsection{System Diagrams}
\begin{frame}
	\frametitle{Connection Diagram}
\end{frame}


\begin{frame}
	\frametitle{Disconnection Diagram}
\end{frame}


\section{Challanges and Obstacles}
\begin{frame}
	\frametitle{Challanges and Obstacles}
    \begin{itemize}
    	\item Installing rabbitmq on Arch
    	\item Determening rabbitmq function error types
    	\item Getting python's dictionaries within dictionaries to work right
	\end{itemize}
\end{frame}


\section{Overcoming !!}
\begin{frame}
	\frametitle{Overcoming !!}
    \begin{itemize}
    	\item Installing rabbitmq on Arch $\rightarrow$ Hacked away at it
    	\item Determening rabbitmq function error types $\rightarrow$ Trial and error
    	\item Getting python's dictionaries within dictionaries to work right $\rightarrow$ Initialize before using
	\end{itemize}
\end{frame}

\end{document}
